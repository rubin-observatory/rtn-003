\documentclass[DM,authoryear,toc]{lsstdoc}
% lsstdoc documentation: https://lsst-texmf.lsst.io/lsstdoc.html
\input{meta}

% Package imports go here.
\usepackage{pdflscape}
\usepackage{afterpage}

% Local commands go here.

%If you want glossaries
\input{aglossary.tex}
\makeglossaries

\title[Independent DACs]{Guidelines for Rubin Independent Data Access Centers}

% Optional subtitle
% \setDocSubtitle{A subtitle}

\author   {William O'Mullane, Beth Willman, Melissa Graham, Leanne Guy, Robert Blum, Phil Marshall }


\setDocRef{RTN-003}
\setDocUpstreamLocation{\url{https://github.com/lsst/rtn-003/}}

\date{\vcsDate}

% Optional: name of the document's curator
% \setDocCurator{The Curator of this Document}

\setDocAbstract{%
This document provides guidelines for groups that are independent from the Rubin Project and Operations (i.e. US Data Facility) and would like to stand up an independent Data Access Center (IDAC; existing data centers that could serve Rubin data products are considered IDACs for purposes of this document). Some IDACs may want to serve only a subset of the  data products: this document proposes three portion sizes, from full releases to a "lite" catalog without posteriors. Guidelines and requirements for IDACs in terms of data storage, computational resources, dedicated personnel, and user authentication are described, as well as a preliminary assessment of the cost impacts. Some institutions, even those inside the US and Chile, may serve LSST data products locally to their research community. Requirements and responsibilities for such institutional bulk data transfers are also described here.
}

% Change history defined here.
% Order: oldest first.
% Fields: VERSION, DATE, DESCRIPTION, OWNER NAME.
% See LPM-51 for version number policy.
\setDocChangeRecord{%
  \addtohist{1}{2020-06-22}{Migrated document from LPM-251}{Leanne Guy}
  \addtohist{1.1}{2020-08-05}{Fix some inconsistencies }{WOM}
  \addtohist{1.2}{2020-10-20}{Fix some more inconsistencies }{WOM}
  \addtohist{1.3}{2021-06-29}{Rubin template, minor typos, update dmtn-135 tables }{WOM}
}


\begin{document}

% Create the title page.
\maketitle
% Frequently for a technote we do not want a title page  uncomment this to remove the title page and changelog.
% use \mkshorttitle to remove the extra pages

\renewcommand{\thepage}{\arabic{page}}% Arabic numerals for page counter

\setcounter{page}{1}% Start page number

% ADD CONTENT HERE
% You can also use the \input command to include several content files.

\section{Introduction}\label{sec:intro}


The current model for \VRO is that is provides proprietary data to approved users in Chile and the \gls{US}. The data access model accommodates this restricted data rights policy. This policy requires control over access, publication, and sharing of proprietary data which any (\gls{IDAC}) would have to comply with just as the \gls{US} and Chile DACs do.

Access to \RO data products for any users will be possible through a \gls{DAC}. The United States's \gls{DAC} , referred to as the \gls{US} Data Facility, is
where registered \RO~ users will perform scientific queries. Most users will have access to a default set of resources at the \gls{DAC} sufficient for basic queries and analysis. Users who require more resources will be able to apply for them, and those granted additional resources will be allowed (for example) to perform analysis on the full data releases using the \gls{RSP}. The \gls{RSP} is documented with the vision given in \citeds{LSE-319}, with more formal requirements in \citeds{LDM-554} and the design in \citeds{LDM-542}. The Chilean \gls{DAC} will be equivalent in functionality to the \gls{US} \gls{DAC}, but scaled-down in terms of the computational resources available for query and analysis given the smaller Chilean community \citedsp{LDM-572}.

%This document proposes a set of guidelines and policies for partner institutions -- in the US, Chile, or one of the International Contributors with signed Memoranda of Agreement -- that are interested in hosting the LSST data, in whole or in part, for their affiliated members as an independent Data Access Center (IDAC).
The following sections include the types of data products that could be hosted (Section \ref{sec:data}), the requirements and responsibilities that would be expected of an \gls{IDAC} hosting \RO proprietary data products (Section \ref{sec:reqs}), and a description of the main costs {\it vs.} their science impacts (Section \ref{sec:costs}).

The contents of this draft document are meant to provide a preliminary resource for partner institutions who may be assessing the feasibility of hosting an \gls{IDAC}. The specific mechanisms and processes by which future \gls{IDAC}s will negotiate the bulk transfer of data, the installation of software, etc. is considered beyond the scope of this document. A simplified checklist is given in \appref{sec:checklist}.

To better understand the sizes of \RO data products, \tabref{tab:storageSizingOps}  gives an overview  of sizes and
the estimaed storage needs are in \tabref{tab:storageFloorOps}(from \citeds{DMTN-135}).


\begin{landscape}
\input{dmtn-135/storageSizingOps.tex}
\input{dmtn-135/storageFloorOps.tex}

\end{landscape}

All access to, and use of the \RO data and data products is subject to the policies described in \citeds{LDO-13}.

In addition to the sizes shown in \tabref{tab:storageSizingOps} it is interesting to consider how much access and potentially how much science
there is per table. This is discussed in detail in \citeds{PSTN-003}. The \gls{AMCL} made an interesting table concerning this topic
which is reproduced here in \tabref{tab:use}. Feedback on the correctness of this table has been sought from \gls{PST}.
\input{images/usetab.tex} % this is in the images git repo


%\input{public}
\input{dataproducts}
\section{Requirements and Guidelines for IDACs}\label{sec:reqs}
Since creating, delivering, and supporting the implementation of \RO data products via IDACs creates some cost to the \RO Project, IDACs will be expected to follow some basic requirements and guidelines, which are described below.
The actual costs of \gls{IDAC} support and infrastructure development are considered separately in Section \ref{sec:xfercost}.
We should also consider that there is the possibility of a {\it lite} DAC or a {\it full} DAC.


\subsection{\RO site topology} \label{sec:topology}

Figure \ref{fig:idac-topology} shows a  topology for a set of interconnected IDACs.  \gls{US} scientists will have direct access to the \RO \gls{US} Data Access Facility.  Hosting on the cloud is shown.

\begin{figure}
\begin{center}
\includegraphics[width=0.95\textwidth]{images/idac-topology}
\caption{US Data Facility and \gls{IDAC} network topology.  \label{fig:idac-topology}}
\end{center}
\end{figure}

\subsection{Authentication and Access}\label{sec:auth}
Any DAC will have to support authentication according to \RO access rules. This may imply delegating access control to a \RO authorization service. In addition any user with access rights should be allowed access to the IDAC.

We should be clear that on the construction side we have not necessarily planned to Authenticate users at various IDACs so there may be some development work needed to make an Auth service available.

\subsection{Required Resources} \label{sec:resources}
Institutions or organizations wishing to set up independent data access centres will be expected to have
sufficient resources and commitments before we discuss data transfers and support.
See also \secref{sec:cvs} for a discussion on compute vs storage.
The exact commitment of course depends on the level of IDAC being implemented.

\subsubsection{Data Storage}
Any institution considering setting up an \gls{IDAC} will need to show commitment on purchasing sufficient storage and \gls{CPU} power to hold and serve the data. Sufficient storage ranges from $0.5$ exabytes for the full data release(s) down to $100$ terabytes for a catalog server, and potentially further down to $70$ terabytes if the {\tt Object Lite} option is offered. For the full catalog , of order 100 nodes are required to serve it up. To serve images, a \gls{DAC} would need some additional servers; depending on load this may be order 10 additional nodes.

\subsubsection{Dedicated Personnel}
The significant hardware required by a full  \gls{IDAC} is above the normal level for most astronomy departments, and would require dedicated technical personnel to set it up and keep it running. For an {\tt Object Lite} catalog running on existing hardware, this might not be a significant increase in person power if the hardware is already serving on order $50$--$100$ terabytes. Still, it is recommended to assume $\gtrsim0.25$ full-time equivalent (\gls{FTE}) personnel hours for {\tt Object Lite}, and perhaps closer to $\sim2$ \gls{FTE} for the full catalogs, which includes setting up and maintaining the service, and installing new data releases and software updates every year. For IDACs wishing to host the full data releases' images and catalogs and deploy the \gls{RSP}, it becomes necessary to employ $1$--$2$ storage engineers to mange the large amount of data, and possibly one more \gls{FTE} to keep the \gls{Kubernetes} (or equivalent) system updated with the latest software deploys. If the \gls{IDAC} intends to support the science of many local users, support will become a specific issue which may not be covered by the usual institutional funding, and will require further effort. It is therefore recommended that any partner institution wishing to host a full-release \gls{IDAC} provide a minimum personnel of 5 \gls{FTE} to be considered viable.

\subsection{Networking and distribution}
There is an assumption than any prospective \gls{IDAC} will have a high bandwidth connection.
Any full \gls{IDAC} should  demonstrate sustained $20 \gls{Gb}/s$ to enable data transfer and sufficient bandwidth for access by users.
In addition all IDACs should be ready  to serve the {\tt \gls{Object} Lite} catalog to any institution worldwide but especially any {\em local} institutions. This should be thought of as a redistribution mechanism for the catalogs.

\subsection{Services for Full IDACs}

Full IDACs will be expected to provide services analogous to those provided at the \gls{US} Data Facility.

\subsubsection{The Rubin \gls{Science Platform}}

The {\it Rubin \gls{Science Platform}}  is a set of integrated web applications and services deployed at  \RO Data Access Centers through which the scientific community will access, visualize, subset and perform next-to-the-data analysis of the data collected by the \RO; it is envisioned to enable science cases that would greatly benefit from the co-location of user processing and \gls{LSST} data. It will provide users access to the {\tt Data Products} described in \ref{sec:data}, such as, resources for image reprocessing, access to the \RO processing framework, and many other services as described in \citeds{LSE-163}.  All full \gls{IDAC} will be expected to run and support the \gls{RSP}.

The \gls{RSP} is described in full in \citeds{LSE-319} and \citeds{LDM-554}. Depending on the assumed load, the \gls{RSP} is relatively modest as it requires only $\sim2$ servers to set up, and it is recommended to have 2 CPUs per simultaneous user (e.g., if the \gls{IDAC}'s desired capability is to serve 200 users, but only expect 50 to be active at a time, then 100 CPUs would be sufficient). From that starting point, the amount of next-to-the-data computational resources can be as large as the data center wishes to provide, and may make use of connecting to e.g., local super computer resources.
\subsubsection{User Generated data products }

{\it User Generated} data products will be created by the community deriving from the {\it Prompt} and {\it \gls{Data Release}} data products, and making  full use of the power of the \RO database systems and
Science Platform for the storage, access, and analysis of their results.
The \gls{Science Platform} will allow for the creation of {\it User Generated} data products and will enable science cases that greatly benefit from co-location of user processing and/or data within the USDF. Full IDACs will be expected to provide support for the creation of {\it User Generated} data products and their federation with the \gls{LSST} Data products.

\section{Responsibilities of the \gls{US} Data Facility}

This section describes the services that the USDF will provide in support of all  IDACs.

The \gls{LDF} will prepare data products for distribution to IDACs along with documentation of hardware and software that will make serving \RO data consistent with the serving of data from the USDF. \RO will provide (modest) technical support consistent with available resources to assist groups setting up IDACs.

LSST, through the USDF will establish a process for potential \gls{IDAC} groups to interface with and establish data transfers to their IDACs. It is expected that \gls{IDAC} groups will propose to \gls{LSST} what their \gls{IDAC} would support and then \gls{LSST} will work with them to establish requirements to receive \gls{LSST} data. One approved, \gls{LSST} will provide (modest) technical support consistent with available resources to assist groups setting up their IDACs provided they comply with prerequisites discussed in this document and especially in \secref{sec:resources}.

\subsection{Data Distribution} \label{sec:dist}

USDF will have $100Gb/s$ connections  on \gls{ESNet} which has interconnects with Internet2 - this should provide a distribution mechanism for getting data to IDACs, it will however be limited by the fact that much of our bandwidth is already allocated for data transmission to \gls{IN2P3} and alert distribution.

A tiered model as used by \gls{CERN} for high energy physics would seem a desirable way to achieve big transfers. Hence we would have a small selection of tier 2 centres with all data products from which tier 3 centres could copy the subsets they wish to work with.  Other alternatives are discuss in \secref{sec:xfer}.

In \gls{HEP} experiments such as  BaBar various physics analysis groups (science collaborations in \gls{LSST} ) were assigned to specific international centers as their primary computing and analysis facility, thereby distributing the computing load around the "network." Users naturally tend to use the facility with available resources and cycles.

As of 2021 the baseline for large transfers looks like Rucio\footnote{\url{https://rucio.cern.ch/}}.



\section{Cost Impacts}\label{sec:costs}

As previously mentioned, standing up and maintaining multiple IDACs comes at a significant cost impact to both the \RO Project and the partner institutions. Minimizing these costs -- or at least maximizing the amount of science they enable -- should be at the forefront of all considerations concerning partner IDACs, such as the following propositions.

\subsection{Maximizing Profits with Science-Driven IDACs}
There are two main cost impacts of IDACs being set up outside of the \gls{US} and Chilean DACs: the positive impact is that some computational load may be taken off of these existing DACs, but the negative impact is the level of support required from the \RO Project in order to get them set up and running. This negative impact could be mitigated by ensuring that science productivity is maximized as a result of this extended effort. One way to do this might be to associate specific areas of science to a given \gls{IDAC}, and encourage users working in that field to use that \gls{IDAC}. This could create a customer base for the \gls{IDAC}, bring together like-minded experts, and effectively distribute the computing load across a network of IDACs. This might also enhance internal funding arguments for investment resources by arguing for synergies with local science goals and attracting international users and official endorsement.

\subsection{Data Transfer}\label{sec:xfer}
Even with good networks the data transfer will not be trivial, and could be quite expensive. \RO is not currently set up to distribute data to multiple sites, i.e., there is no form of peer-to-peer sharing. The bandwidth at \gls{USDF} is adequate for receiving data and delivering {\tt Alerts} to brokers during the night; perhaps some day time bandwidth could be used to transfer data to IDACs. A full data release of images and catalogs does not have to transferred within a given day; if the correct agreements are in place with an \gls{IDAC}, a full release could be transferred slowly as it is produced, and then made available to the IDACs users in whole on the official release day.
\input{xfercost}

\subsection{Compute {\it vs.} Storage Resources} \label{sec:cvs}
Data storage could be a large cost to IDACs, and could be considered as an overhead relative to the amount of computational resources an \gls{IDAC} can offer. If a full \gls{IDAC} is set up without a large compute capacity, the facility might be less useful to the science community than e.g., augmenting an existing DAC or \gls{IDAC} to have more computational resources. It is conceivable that a partner institution may prefer to spend their money increasing the computational quotas available for a given collaboration or set of PIs, and it would be scientifically beneficial if this was possible at all DAC and IDACs. The notion of standard compute quotas and resource allocation committees to adjudicate on large proposals for substantial increases to computational allocations are described in \citeds{LDO-13}. Another way to approach a solution to this issue might be to have a \emph{Cloud}-based \gls{IDAC} where a user or \gls{PI} could buy nodes on the provider cloud to access the holdings put there by \RO.  Such an option may be particularly useful to Science Collaborations with large compute needs.

The full sizing model is in \citeds{DMTN-135} - any \gls{IDAC} should have a similar sizing model. They may not need as much compute or as many copies of data as we have but the raw information to make such calls are in the technote.  For ball park figures the construction
to first year of ops table is copied here as \tabref{tab:opsSumUSDF}\footnote{ There is no guarantee of being in sync with \citeds{DMTN-135} but as an order of magnitude it is good.}.

\input{summary}



\appendix
% Include all the relevant bib files.
% https://lsst-texmf.lsst.io/lsstdoc.html#bibliographies
\section{References} \label{sec:bib}
\renewcommand{\refname}{} % Suppress default Bibliography section
\bibliography{local,lsst,lsst-dm,refs_ads,refs,books}

% Make sure lsst-texmf/bin/generateAcronyms.py is in your path
%\section{Acronyms}
%\input{acronyms.tex}
% If you want glossary uncomment below -- comment out the two lines above
\label{sec:acronyms}
\printglossaries


\input{checklist}
\end{document}
