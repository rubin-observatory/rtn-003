\documentclass[DM,authoryear,toc]{lsstdoc}
% lsstdoc documentation: https://lsst-texmf.lsst.io/lsstdoc.html
\input{meta}

% Package imports go here.
\usepackage{pdflscape}
\usepackage{afterpage}

% Local commands go here.

%If you want glossaries
\input{aglossary.tex}
\makeglossaries

\title[Independent DACs]{Guidelines for Rubin Independent Data Access Centers}

% Optional subtitle
% \setDocSubtitle{A subtitle}

\author   {William O'Mullane, Beth Willman, Melissa Graham, Leanne Guy, Robert Blum, Phil Marshall }


\setDocRef{RTN-003}
\setDocUpstreamLocation{\url{https://github.com/lsst/rtn-003/}}

\date{\vcsDate}

% Optional: name of the document's curator
% \setDocCurator{The Curator of this Document}

\setDocAbstract{%
This document provides guidelines for groups that are independent from the Rubin Project and Operations (i.e. US Data Facility) and would like to stand up an independent Data Access Center (IDAC; existing data centers that could serve Rubin data products are considered IDACs for purposes of this document). Some IDACs may want to serve only a subset of the  data products: this document proposes three portion sizes, from full releases to a "lite" catalog without posteriors. Guidelines and requirements for IDACs in terms of data storage, computational resources, dedicated personnel, and user authentication are described, as well as a preliminary assessment of the cost impacts. Some institutions, even those inside the US and Chile, may serve LSST data products locally to their research community. Requirements and responsibilities for such institutional bulk data transfers are also described here.
}

% Change history defined here.
% Order: oldest first.
% Fields: VERSION, DATE, DESCRIPTION, OWNER NAME.
% See LPM-51 for version number policy.
\setDocChangeRecord{%
  \addtohist{1}{2020-06-22}{Migrated document from LPM-251}{Leanne Guy}
  \addtohist{1.1}{2020-08-05}{Fix some inconsistencies }{WOM}
  \addtohist{1.2}{2020-10-20}{Fix some more inconsistencies }{WOM}
  \addtohist{1.3}{2021-06-29}{Rubin template, minor typos, update dmtn-135 tables }{WOM}
}


\begin{document}

% Create the title page.
\maketitle
% Frequently for a technote we do not want a title page  uncomment this to remove the title page and changelog.
% use \mkshorttitle to remove the extra pages

\renewcommand{\thepage}{\arabic{page}}% Arabic numerals for page counter

\setcounter{page}{1}% Start page number

% ADD CONTENT HERE
% You can also use the \input command to include several content files.

\section{Introduction}\label{sec:intro}


The current model for \VRO is that is provides proprietary data to approved users in Chile and the \gls{US}. The data access model accommodates this restricted data rights policy. This policy requires control over access, publication, and sharing of proprietary data which any (\gls{IDAC}) would have to comply with just as the \gls{US} and Chile DACs do.

Access to \RO data products for any users will be possible through a \gls{DAC}. The United States's \gls{DAC} , referred to as the \gls{US} Data Facility, is
where registered \RO~ users will perform scientific queries. Most users will have access to a default set of resources at the \gls{DAC} sufficient for basic queries and analysis. Users who require more resources will be able to apply for them, and those granted additional resources will be allowed (for example) to perform analysis on the full data releases using the \gls{RSP}. The \gls{RSP} is documented with the vision given in \citeds{LSE-319}, with more formal requirements in \citeds{LDM-554} and the design in \citeds{LDM-542}. The Chilean \gls{DAC} will be equivalent in functionality to the \gls{US} \gls{DAC}, but scaled-down in terms of the computational resources available for query and analysis given the smaller Chilean community \citedsp{LDM-572}.

%This document proposes a set of guidelines and policies for partner institutions -- in the US, Chile, or one of the International Contributors with signed Memoranda of Agreement -- that are interested in hosting the LSST data, in whole or in part, for their affiliated members as an independent Data Access Center (IDAC).
The following sections include the types of data products that could be hosted (Section \ref{sec:data}), the requirements and responsibilities that would be expected of an \gls{IDAC} hosting \RO proprietary data products (Section \ref{sec:reqs}), and a description of the main costs {\it vs.} their science impacts (Section \ref{sec:costs}).

The contents of this draft document are meant to provide a preliminary resource for partner institutions who may be assessing the feasibility of hosting an \gls{IDAC}. The specific mechanisms and processes by which future \gls{IDAC}s will negotiate the bulk transfer of data, the installation of software, etc. is considered beyond the scope of this document. A simplified checklist is given in \appref{sec:checklist}.

To better understand the sizes of \RO data products, \tabref{tab:storageSizingOps}  gives an overview  of sizes and
the estimaed storage needs are in \tabref{tab:storageFloorOps}(from \citeds{DMTN-135}).


\begin{landscape}
\input{dmtn-135/storageSizingOps.tex}
\input{dmtn-135/storageFloorOps.tex}

\end{landscape}

All access to, and use of the \RO data and data products is subject to the policies described in \citeds{LDO-13}.

In addition to the sizes shown in \tabref{tab:storageSizingOps} it is interesting to consider how much access and potentially how much science
there is per table. This is discussed in detail in \citeds{PSTN-003}. The \gls{AMCL} made an interesting table concerning this topic
which is reproduced here in \tabref{tab:use}. Feedback on the correctness of this table has been sought from \gls{PST}.
\input{images/usetab.tex} % this is in the images git repo


%\section{Sizing the Data Model}\label{sec:public}

LSST must support a large number of users. The original data model assumed a certain distribution of queries and it is useful to consider building more margin  into the data centers by providing a range of data products to serve different needs. Not all users will need to access all the data.

Alex Szalay (ref) noted there were over a million unique IP addresses which hit the SDSS archive over a one year period.  Gaia saw 2K users accounts per hour on the catalogue release. These numbers suggest significant public interest beyond professionals, and this load must be supported by the DAC as well. There are several possible ways to handle the load. 
\footnote{Gaia reports modest discontent due to user load \url{https://blogs.scientificamerican.com/observations/the-price-of-open-science/}},

Nationally, we could partner with Google or Amazon who will notionally host public data sets for free - so we could more selectively add users to the DACs and try to put most casual users to the public interface. It is not clear if this would then be the EPO interface, but that is worth considering. EPO would no longer have to select 10\% of data, but it would have a bigger job to deal with a full data release. Still, combining DM and EPO seamlessly with respect to data would be favorable.
The public data set would be some lite version for the catalog \secref{sec:lite}  and the HiPS type color images. No raw data files and potentially no advanced notebook type access. This may not even need a logon for the fast queries like $``$show me M31$''$ with LSST sources plotted. EPO queries would not include the source catalogue which is large ($10^{13}$ rows).
Potential national partners  could also host the object catalog e.g.

\begin{itemize}
 \item MAST at STScI
 \item DataLab at NOAO
 \item SAO
 \item US Naval Observatory
 \item NED at IPAC
 \item HEASARC
 \item CADC - Canadian Astronomical Data Centre - we work with them already in Data Management.
\end{itemize}

\subsection {International}
Internationally we could partner with a network of sites - it should be a network to allow peer to peer sharing of catalogs. Here we could provide the HIPS color images and again some version of the OBJECT catalogue. We would have to consider if the source catalogue should also be distributed to a smaller subset of centres who could cope with it.
Potential International Partners might be:

\begin{itemize}
\item In Europe we have a few centres that Astronomers expect to find sources at:
\begin{itemize}
    \item CDS  - Aladin and Vizier - this is a minimum for Europe
    \item ESAC - ESASky - European Space Agency
    \item  ASDC - Italian Data Center
    \item  GAVO MPA - German Astronomical Virtual Observatory - Max Plank Astrophysics
    \item  IoA - Cambridge
    \item  Edinburgh
\end{itemize}
\item In Asia
\begin{itemize}
    \item  NAOJ Astronomical Data Center, Tokyo
    \item  Chinese Academy of Sciences - perhaps National Space Science Center
\end{itemize}
\item In South America
\begin{itemize}
    \item We already have a Chilean DAC
    \item LIneA in Brazil - we already work with them
\end{itemize}
\end{itemize}

Having the lite catalog hosted at multiple locations where, and in formats which, astronomers would expect to find catalog information would reduce load on the US and Chilean DAC. It will also put the LSST data more readily in the hands of the astronomers and should accelerate science at least in the cases where the catalog is the prime source of information, for example, Galactic dynamics and other large statistical studies.

\section{Types of Data Products for IDACs}\label{sec:data}
% Operations-era language for data products, as defined in  LSE-231, should be used

The three categories of LSST data products, {\it Prompt}, {\it Data Release}, and {\it User Generated} are defined in the LSST Data Product Categories document  \citeds{LPM-231}.
Both the {\it Prompt}, {\it Data Release} data products are produced by LSST and include images, both raw and processed, and catalogs of both {\tt Object}s and {\tt Source}s.
The  {\it User Generated} data products are produced the community using the resources of the LSST Science Platform \citeds{LSE-319}.
These data products are described in detail in the Data Products Definitions Document \citeds{LSE-163}.

Below, three potential realizations of the the LSST {\it Data Release} data products that IDACs might consider hosting are described: the full {\it Data Release} including images, the {\it Data Release} catalogs only, and a low-volume (``{\tt lite}'') subset of the {\it Data Release} catalogs.

\subsection{Full Data Release(s)}

In this case the IDAC would be hosting all of the raw and processed images, and catalogs, as described in \citedsp{LSE-163}. Hosting the raw image data at an IDAC requires roughly $6$ petabytes per year of storage, so this represents a significant augmentation of resources in terms of both hardware and personnel. The processed data and associated calibrations bring the total data volume to $0.5$ exabytes for a single data release. Some data volume could be saved by taking only a single calibrated image per band, but the total would still be $60$ petabytes (with compression it may be possible to reduce this even further). Any IDAC considering hosting the full {\it Data Release} should also deploy the full LSST Science Platform \citeds{LSE-319} in order to maximize science productivity and their return on investment in hosting an IDAC.

\subsection{Catalog Server}

Alternatively, an IDAC may find that hosting only the {\it Data Release} catalogs, and not the images, is sufficient for the scientific needs of its community. This will probably require the specific LSST database server \citedsp{LDM-135} and specific machines, and the deployment of the database system and the associated subset of data access services (DAX; e.g., web APIs, Qserv, \citeds{LDM-152}). The full {\tt Object} catalog, which contains one row per object with a volume of $\approx 20$ kilobytes per row, is estimated to contain about $40 \times 10^9$ objects (even in the first full-sky data release). Adding to this the full {\tt Source} and {\tt Forced Source} catalogs, which contain one row per measurement in each of the $\sim80$ visit images obtained per year, brings the total storage volume required up into the petabytes range, and will require a serious commitment of resources at the proposed IDAC. The evolution of catalog sizes over the 10-year LSST survey is depicted in \figref{fig:catvol}, from which it is evident that the catalog size for the final release is order $15$ petabytes. For more details on the row counts see the Key Numbers Page\footnote{\url{https://confluence.lsstcorp.org/display/LKB/LSST+Key+Numbers}}.

\begin{figure}
\begin{center}
\includegraphics[width=0.8\textwidth]{images/CatVolTime.png}
\caption{Catalog volume over time from \citeds{LDM-144}. \label{fig:catvol}}
\end{center}
\end{figure}

\subsection{An ``{\tt Object Lite}" Catalog}\label{sec:lite}

Many -- perhaps most -- astronomers' science goals will be adequately served by a low-volume subset of the {\tt Object} catalog's columns that do not include, for example, the full posteriors for the bulge+disk likelihood parameters.
This {\tt Object Lite} catalog would nominally contain $1840$ bytes per row for the $40 \times 10^{9}$ objects, giving a size of $\approx 7.4 \times 10^{13}$ bytes ($\sim74$ terabytes).
Even smaller, science-specific versions of {\tt Object Lite} could be envisioned with even less columns and/or separate star and galaxy catalogs.
The Solar System community for example will be primarily be interested in the contents of  just the  {\tt SSObjects} table.

These would not be small enough to handle on a laptop, but might be served by a small departmental cluster.
Searching even a small {\tt Object Lite} catalog would require some form of database, but many institutes would already have a system which may be capable of loading this data.
In this case, LSST might only ship files with documentation and not provide administrative support for the system, but this would allow the {\tt Object Lite} catalog to be widely available to all partner institution IDACs. Distribution options such as peer-to-peer networking to avoid download bandwidth limitations might be possible to implement in this case.
%See also \secref{sec:public}.


\section{Requirements and Guidelines for IDACs}\label{sec:reqs}
Since creating, delivering, and supporting the implementation of \RO data products via IDACs creates some cost to the \RO Project, IDACs will be expected to follow some basic requirements and guidelines, which are described below.
The actual costs of \gls{IDAC} support and infrastructure development are considered separately in Section \ref{sec:xfercost}.
We should also consider that there is the possibility of a {\it lite} DAC or a {\it full} DAC.


\subsection{\RO site topology} \label{sec:topology}

Figure \ref{fig:idac-topology} shows a  topology for a set of interconnected IDACs.  \gls{US} scientists will have direct access to the \RO \gls{US} Data Access Facility.  Hosting on the cloud is shown.

\begin{figure}
\begin{center}
\includegraphics[width=0.95\textwidth]{images/idac-topology}
\caption{US Data Facility and \gls{IDAC} network topology.  \label{fig:idac-topology}}
\end{center}
\end{figure}

\subsection{Authentication and Access}\label{sec:auth}
Any DAC will have to support authentication according to \RO access rules. This may imply delegating access control to a \RO authorization service. In addition any user with access rights should be allowed access to the IDAC.

We should be clear that on the construction side we have not necessarily planned to Authenticate users at various IDACs so there may be some development work needed to make an Auth service available.

\subsection{Required Resources} \label{sec:resources}
Institutions or organizations wishing to set up independent data access centres will be expected to have
sufficient resources and commitments before we discuss data transfers and support.
See also \secref{sec:cvs} for a discussion on compute vs storage.
The exact commitment of course depends on the level of IDAC being implemented.

\subsubsection{Data Storage}
Any institution considering setting up an \gls{IDAC} will need to show commitment on purchasing sufficient storage and \gls{CPU} power to hold and serve the data. Sufficient storage ranges from $0.5$ exabytes for the full data release(s) down to $100$ terabytes for a catalog server, and potentially further down to $70$ terabytes if the {\tt Object Lite} option is offered. For the full catalog , of order 100 nodes are required to serve it up. To serve images, a \gls{DAC} would need some additional servers; depending on load this may be order 10 additional nodes.

\subsubsection{Dedicated Personnel}
The significant hardware required by a full  \gls{IDAC} is above the normal level for most astronomy departments, and would require dedicated technical personnel to set it up and keep it running. For an {\tt Object Lite} catalog running on existing hardware, this might not be a significant increase in person power if the hardware is already serving on order $50$--$100$ terabytes. Still, it is recommended to assume $\gtrsim0.25$ full-time equivalent (\gls{FTE}) personnel hours for {\tt Object Lite}, and perhaps closer to $\sim2$ \gls{FTE} for the full catalogs, which includes setting up and maintaining the service, and installing new data releases and software updates every year. For IDACs wishing to host the full data releases' images and catalogs and deploy the \gls{RSP}, it becomes necessary to employ $1$--$2$ storage engineers to mange the large amount of data, and possibly one more \gls{FTE} to keep the \gls{Kubernetes} (or equivalent) system updated with the latest software deploys. If the \gls{IDAC} intends to support the science of many local users, support will become a specific issue which may not be covered by the usual institutional funding, and will require further effort. It is therefore recommended that any partner institution wishing to host a full-release \gls{IDAC} provide a minimum personnel of 5 \gls{FTE} to be considered viable.

\subsection{Networking and distribution}
There is an assumption than any prospective \gls{IDAC} will have a high bandwidth connection.
Any full \gls{IDAC} should  demonstrate sustained $20 \gls{Gb}/s$ to enable data transfer and sufficient bandwidth for access by users.
In addition all IDACs should be ready  to serve the {\tt \gls{Object} Lite} catalog to any institution worldwide but especially any {\em local} institutions. This should be thought of as a redistribution mechanism for the catalogs.

\subsection{Services for Full IDACs}

Full IDACs will be expected to provide services analogous to those provided at the \gls{US} Data Facility.

\subsubsection{The Rubin \gls{Science Platform}}

The {\it Rubin \gls{Science Platform}}  is a set of integrated web applications and services deployed at  \RO Data Access Centers through which the scientific community will access, visualize, subset and perform next-to-the-data analysis of the data collected by the \RO; it is envisioned to enable science cases that would greatly benefit from the co-location of user processing and \gls{LSST} data. It will provide users access to the {\tt Data Products} described in \ref{sec:data}, such as, resources for image reprocessing, access to the \RO processing framework, and many other services as described in \citeds{LSE-163}.  All full \gls{IDAC} will be expected to run and support the \gls{RSP}.

The \gls{RSP} is described in full in \citeds{LSE-319} and \citeds{LDM-554}. Depending on the assumed load, the \gls{RSP} is relatively modest as it requires only $\sim2$ servers to set up, and it is recommended to have 2 CPUs per simultaneous user (e.g., if the \gls{IDAC}'s desired capability is to serve 200 users, but only expect 50 to be active at a time, then 100 CPUs would be sufficient). From that starting point, the amount of next-to-the-data computational resources can be as large as the data center wishes to provide, and may make use of connecting to e.g., local super computer resources.
\subsubsection{User Generated data products }

{\it User Generated} data products will be created by the community deriving from the {\it Prompt} and {\it \gls{Data Release}} data products, and making  full use of the power of the \RO database systems and
Science Platform for the storage, access, and analysis of their results.
The \gls{Science Platform} will allow for the creation of {\it User Generated} data products and will enable science cases that greatly benefit from co-location of user processing and/or data within the USDF. Full IDACs will be expected to provide support for the creation of {\it User Generated} data products and their federation with the \gls{LSST} Data products.

\section{Responsibilities of the \gls{US} Data Facility}

This section describes the services that the USDF will provide in support of all  IDACs.

The \gls{LDF} will prepare data products for distribution to IDACs along with documentation of hardware and software that will make serving \RO data consistent with the serving of data from the USDF. \RO will provide (modest) technical support consistent with available resources to assist groups setting up IDACs.

LSST, through the USDF will establish a process for potential \gls{IDAC} groups to interface with and establish data transfers to their IDACs. It is expected that \gls{IDAC} groups will propose to \gls{LSST} what their \gls{IDAC} would support and then \gls{LSST} will work with them to establish requirements to receive \gls{LSST} data. One approved, \gls{LSST} will provide (modest) technical support consistent with available resources to assist groups setting up their IDACs provided they comply with prerequisites discussed in this document and especially in \secref{sec:resources}.

\subsection{Data Distribution} \label{sec:dist}

USDF will have $100Gb/s$ connections  on \gls{ESNet} which has interconnects with Internet2 - this should provide a distribution mechanism for getting data to IDACs, it will however be limited by the fact that much of our bandwidth is already allocated for data transmission to \gls{IN2P3} and alert distribution.

A tiered model as used by \gls{CERN} for high energy physics would seem a desirable way to achieve big transfers. Hence we would have a small selection of tier 2 centres with all data products from which tier 3 centres could copy the subsets they wish to work with.  Other alternatives are discuss in \secref{sec:xfer}.

In \gls{HEP} experiments such as  BaBar various physics analysis groups (science collaborations in \gls{LSST} ) were assigned to specific international centers as their primary computing and analysis facility, thereby distributing the computing load around the "network." Users naturally tend to use the facility with available resources and cycles.

As of 2021 the baseline for large transfers looks like Rucio\footnote{\url{https://rucio.cern.ch/}}.



\section{Cost Impacts}\label{sec:costs}

As previously mentioned, standing up and maintaining multiple IDACs comes at a significant cost impact to both the \RO Project and the partner institutions. Minimizing these costs -- or at least maximizing the amount of science they enable -- should be at the forefront of all considerations concerning partner IDACs, such as the following propositions.

\subsection{Maximizing Profits with Science-Driven IDACs}
There are two main cost impacts of IDACs being set up outside of the \gls{US} and Chilean DACs: the positive impact is that some computational load may be taken off of these existing DACs, but the negative impact is the level of support required from the \RO Project in order to get them set up and running. This negative impact could be mitigated by ensuring that science productivity is maximized as a result of this extended effort. One way to do this might be to associate specific areas of science to a given \gls{IDAC}, and encourage users working in that field to use that \gls{IDAC}. This could create a customer base for the \gls{IDAC}, bring together like-minded experts, and effectively distribute the computing load across a network of IDACs. This might also enhance internal funding arguments for investment resources by arguing for synergies with local science goals and attracting international users and official endorsement.

\subsection{Data Transfer}\label{sec:xfer}
Even with good networks the data transfer will not be trivial, and could be quite expensive. \RO is not currently set up to distribute data to multiple sites, i.e., there is no form of peer-to-peer sharing. The bandwidth at \gls{USDF} is adequate for receiving data and delivering {\tt Alerts} to brokers during the night; perhaps some day time bandwidth could be used to transfer data to IDACs. A full data release of images and catalogs does not have to transferred within a given day; if the correct agreements are in place with an \gls{IDAC}, a full release could be transferred slowly as it is produced, and then made available to the IDACs users in whole on the official release day.
\subsubsection{Transfer cost use case \label{sec:xfercost}}
If we take the final number from the key numbers page \footnote{\url{https://confluence.lsstcorp.org/display/LKB/LSST+Key+Numbers}} we could consider DR1 as about 6 PB (10\% of the final size).

We would have at least  two ways to transfer this : via the network, via physical devices.

A network transfer at 10Gbps of 6 PB would take $8 * 6 \times 10^{12} \ 10^7 = 4.8 \times 10^{6}~seconds $ or about 55 days\footnote{ day = 86400}.
Many institutes have 100 Gbps connections so this should be an upper limit and a transfer should be order one week. If we had a peer to peer network this may go down somewhat and we may be able to support it from NCSA.

Alternatively we could host the data on Amazon or Google and let people download it from there - they would have more capacity.
Storage on the cloud  for public data would be theoretically free - download (egress)  would  cost.
Transfer to another cloud \footnote{\url{https://cloud.google.com/storage/pricingi\#network-pricing}}  or
 a Content Delivery Network (CDN)\footnote{\url{https://cloud.google.com/cdn/pricing}}
 end up costing  about a cent a GB which for an open science project and at our volume  should  be negotiable.  At one cent a transfer would cost
  $\sim \$0.01 * 6 \times 10^{12} \ 10^6 = \$60K$.

For physical devices, today apparently we could get a device like Petarack \url{https://www.aberdeeninc.com/petarack/} for \$300K.
Theoretically we could get this cheaper though this is close to the drive price,
Tape may also be a possibility especially if Sony/IBM commercialize high density tape with >300TB per cartridge\footnote{\url{https://newatlas.com/sony-ibm-magnetic-tape-density-record/50743/}}. A current 6TB cartridge is about \$30, so enough tapes for 6PB would cost about  30K. If the density increased this could come down significantly.
This could be be a partner data center cost as well as shipping it. Transfer of data on to this would be about the same as the network rate above so 7 days. SneakerNet \cite{2002cs........8011G} may still be  cost effective in the LSST era, which is predicted in the a paper.


\subsection{Compute {\it vs.} Storage Resources} \label{sec:cvs}
Data storage could be a large cost to IDACs, and could be considered as an overhead relative to the amount of computational resources an \gls{IDAC} can offer. If a full \gls{IDAC} is set up without a large compute capacity, the facility might be less useful to the science community than e.g., augmenting an existing DAC or \gls{IDAC} to have more computational resources. It is conceivable that a partner institution may prefer to spend their money increasing the computational quotas available for a given collaboration or set of PIs, and it would be scientifically beneficial if this was possible at all DAC and IDACs. The notion of standard compute quotas and resource allocation committees to adjudicate on large proposals for substantial increases to computational allocations are described in \citeds{LDO-13}. Another way to approach a solution to this issue might be to have a \emph{Cloud}-based \gls{IDAC} where a user or \gls{PI} could buy nodes on the provider cloud to access the holdings put there by \RO.  Such an option may be particularly useful to Science Collaborations with large compute needs.

The full sizing model is in \citeds{DMTN-135} - any \gls{IDAC} should have a similar sizing model. They may not need as much compute or as many copies of data as we have but the raw information to make such calls are in the technote.  For ball park figures the construction
to first year of ops table is copied here as \tabref{tab:opsSumUSDF}\footnote{ There is no guarantee of being in sync with \citeds{DMTN-135} but as an order of magnitude it is good.}.

\tiny \begin{longtable} { |p{0.22\textwidth}  |r  |r  |r  |r  |r  |r  |r  |r  |r  |r  |r |}
\caption{Operations costs summary table from DMTN-135
 \label{tab:opsSumUSDF}}\\
\hline
\textbf{Year  (all prices Million\$)}&\textbf{2023}&\textbf{2024}&\textbf{2025}&\textbf{2026}&\textbf{2027}&\textbf{2028}&\textbf{2029}&\textbf{2030}&\textbf{2031}&\textbf{2032} \\ \hline
{Compute (2019 pricing)}&{\$2.79}&{\$2.95}&{\$4.87}&{\$6.12}&{\$6.28}&{\$7.10}&{\$6.66}&{\$6.66}&{\$7.10}&{\$6.66} \\ \hline
{Qserv (2019 pricing)}&{\$1.62}&{\$2.42}&{\$1.86}&{\$2.40}&{\$2.74}&{\$3.60}&{\$2.10}&{\$2.18}&{\$2.78}&{\$3.12} \\ \hline
{Storage (2019 pricing)}&{\$6.41}&{\$7.46}&{\$8.79}&{\$9.12}&{\$10.72}&{\$15.71}&{\$16.83}&{\$18.17}&{\$18.50}&{\$19.16} \\ \hline
\textbf{Total (2019 pricing)}&\textbf{\$10.82}&\textbf{\$12.83}&\textbf{\$15.52}&\textbf{\$17.64}&\textbf{\$19.74}&\textbf{\$26.41}&\textbf{\$25.59}&\textbf{\$27.01}&\textbf{\$28.38}&\textbf{\$28.94} \\ \hline
{Applying price factor (CPU)}&{\$1.83}&{\$1.74}&{\$2.59}&{\$2.93}&{\$2.70}&{\$2.75}&{\$2.32}&{\$2.09}&{\$2.01}&{\$1.69} \\ \hline
{IN2P3 (50\% of compute)}&{-\$0.92}&{-\$0.87}&{-\$1.29}&{-\$1.46}&{-\$1.35}&{-\$1.38}&{-\$1.16}&{-\$1.04}&{-\$1.00}&{-\$0.85} \\ \hline
{Qserv (applying factor)}&{\$1.19}&{\$1.64}&{\$1.17}&{\$1.39}&{\$1.47}&{\$1.78}&{\$0.96}&{\$0.92}&{\$1.09}&{\$1.13} \\ \hline
{Applying price factor (Storage)}&{\$5.22}&{\$5.77}&{\$6.46}&{\$6.37}&{\$7.11}&{\$9.90}&{\$10.08}&{\$10.33}&{\$9.99}&{\$9.84} \\ \hline
{Hosting Overhead NCSA
}&{\$0.54}&{\$0.79}&{\$1.01}&{\$1.21}&{\$1.38}&{\$1.61}&{\$1.71}&{\$1.85}&{\$2.01}&{\$2.23} \\ \hline
\textbf{Total budget (using price factors)}&\textbf{\$7.86}&\textbf{\$9.08}&\textbf{\$9.93}&\textbf{\$10.43}&\textbf{\$11.31}&\textbf{\$14.67}&\textbf{\$13.91}&\textbf{\$14.16}&\textbf{\$14.10}&\textbf{\$14.04} \\ \hline
\textbf{Total Operations hardware to 2032 }&\textbf{\$119.48}&\textbf{million}&&&&&&&& \\ \hline
\end{longtable} \normalsize




\appendix
% Include all the relevant bib files.
% https://lsst-texmf.lsst.io/lsstdoc.html#bibliographies
\section{References} \label{sec:bib}
\renewcommand{\refname}{} % Suppress default Bibliography section
\bibliography{local,lsst,lsst-dm,refs_ads,refs,books}

% Make sure lsst-texmf/bin/generateAcronyms.py is in your path
%\section{Acronyms}
%\addtocounter{table}{-1}
\begin{longtable}{p{0.145\textwidth}p{0.8\textwidth}}\hline
\textbf{Acronym} & \textbf{Description}  \\\hline

AMCL & AURA Management Council for LSST \\\hline
ASDC & ASI Science Data Center (Italy) \\\hline
CADC & Canadian Astronomy Data Centre \\\hline
CAOM & Common Archive Observation Model \\\hline
CDN & Content Delivery Network \\\hline
CDS & Centre de Donnes astronomiques de Strasbourg \\\hline
CERN & European Organization for Nuclear Research \\\hline
CPU & Central Processing Unit \\\hline
DAC & Data Access Center \\\hline
DAX & Data Access Services \\\hline
DM & Data Management \\\hline
DMTN & DM Technical Note \\\hline
DR1 & Data Release 1 \\\hline
EPO & Education and Public Outreach \\\hline
ESAC & European Space Astronomy Centre \\\hline
ESNet & Energy Sciences Network \\\hline
FTE & Full-Time Equivalent \\\hline
GAVO & German Astronomical Virtual Observatory \\\hline
GB & Gigabyte \\\hline
Gb & Gigabit \\\hline
HEASARC & NASA's Archive of Data on Energetic Phenomena \\\hline
HEP &  High Energy Physics \\\hline
HIPS & Hierarchical Progressive Survey \\\hline
IBM & International Business Machines \\\hline
IDAC & Independent Data Access Center \\\hline
IN2P3 & Institut National de Physique Nucléaire et de Physique des Particules \\\hline
IP & Internet Protocol \\\hline
IPAC & No longer an acronym; science and data center at Caltech \\\hline
IVOA & International Virtual-Observatory Alliance \\\hline
IoA & Institute of Astronomy (Cambridge; also denoted IOA) \\\hline
LDF & LSST Data Facility \\\hline
LDM & LSST Data Management (Document Handle) \\\hline
LDO & LSST Document Operations (Document Handle) \\\hline
LPM & LSST Project Management (Document Handle) \\\hline
LSE & LSST Systems Engineering (Document Handle) \\\hline
LSP & LSST Science Platform \\\hline
LSST & Legacy Survey of Space and Time (formerly Large Synoptic Survey Telescope) \\\hline
MAST & Mikulski Archive for Space Telescopes \\\hline
MPA & Max Planck Institute for Astrophysics \\\hline
NAOJ & National Astronomical Observatory of Japan \\\hline
NCSA & National Center for Supercomputing Applications \\\hline
NED & NASA/IPAC Extragalactic Database \\\hline
NOAO & National Optical Astronomy Observatories (USA) \\\hline
PB & PetaByte \\\hline
PI & Principle Investigator \\\hline
PST & Project Science Team \\\hline
PSTN & Project Science Technical Note \\\hline
SAO & Smithsonian Astrophysical Observatory \\\hline
SDSS & Sloan Digital Sky Survey \\\hline
TAP & Table Access Protocol \\\hline
TB & TeraByte \\\hline
US & United States \\\hline
\end{longtable}

% If you want glossary uncomment below -- comment out the two lines above
\label{sec:acronyms}
\printglossaries


\newlist{todolist}{itemize}{2}
\setlist[todolist]{label=$\square$}

\section{IDAC Proposal Checklist}\label{sec:checklist}

There is a spectrum of possibilities for an \gls{IDAC}'s scope, from just hosting the \gls{Object} lite database, to serving full copies of the current data release and prompt products database.
Here an attempt is made to have a set of check lists  that can be used to look at an \gls{IDAC}.
\secref{sec:fullDAC} covers a full capability \gls{IDAC}, while  \secref{sec:liteDAC} gives the criteria for a minimum capability \gls{IDAC}.
There will undoubtedly be proposals in between. There are some criteria that any \gls{IDAC} must meet, in order to comply with Rubin Observatory data policy. These are given in \secref{sec:anyDAC}.

\subsection{Any \gls{DAC}} \label{sec:anyDAC}
\begin{todolist}
\item Authentication/Authorization system  inline with Rubin Observatory Access
\item Agreement to make broadly accessible to all Data Rights holders
\end{todolist}

\subsection{Lite \gls{IDAC}} \label{sec:liteDAC}
All criteria in \secref{sec:anyDAC}, and then, in addition:
\begin{todolist}
\item Database system capable of handling $4^{10}$ rows.
\item \gls{IVOA} \gls{TAP} interface, MyDB and Table Upload, \gls{CAOM} support.
\item About 500TB of disk for catalogs + MyDBs.
\item Professional support staff (min 0.25 \gls{FTE})
\item Sufficient connectivity to support users
\end{todolist}

\subsection{Full \gls{DAC}} \label{sec:fullDAC}
All criteria in \secref{sec:anyDAC}, and then, in addition:
\begin{todolist}
\item Staff (about 5 \gls{FTE}) to handle major hardware installation
\item Agreement to stand up standard \gls{Science Platform} (Puppet/Kubernetes etc.)
\item Database system capable of handling all catalogs (or \gls{Qserv}) with \gls{IVOA}  interfaces.
\item Understanding of sizing model in \citeds{DMTN-135} - sizing model and a cost model for the \gls{IDAC}.
\item "Commitment to fund the \gls{IDAC} through the \gls{LSST} operations period, FY22-FY34 (probably  $>\$6M/year$ based on hardware cost model and labor plan).".
\item Sufficient connectivity to support data transfer in and user access out at least 20Gbps of free bandwidth.
\end{todolist}

\end{document}
